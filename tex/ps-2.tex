\documentclass[11pt, preprint]{article}
\usepackage{hyperref}
\usepackage{rotating}
 
\setlength{\footnotesep}{9.6pt}
\setlength{\parskip}{12pt}

\newcounter{thefigs}
\newcommand{\fignum}{\arabic{thefigs}}

\newcounter{thetabs}
\newcommand{\tabnum}{\arabic{thetabs}}

\newcounter{address}

\begin{document}

\begin{center}
  {\bf Radiative Processes in Astrophysics / Problem Set \#2 \\
    Due February 24, 2021}
\end{center}

\noindent Model an interstellar cloud of gas and dust as a uniform,
  plane-parallel slab 100 pc thick, with a temperature of 50 K, and
  density dominated by molecular hydrogen with $n \sim 10$
  cm$^{-3}$. (Problem from Aaron Parsons' notes).
\begin{enumerate}
\item Dust is typically made of silicate grains with $\rho \sim 3$ g
  cm$^{-3}$, $r\sim 0.1$ $\mu$ and with a mass fraction of 0.01. What
  is the number density of the dust grains?
\item Imagine a backlight with $I_\nu = 3\times 10^{-9}$
  erg~s$^{-1}$~Hz$^{-1}$~ster$^{-1}$~cm$^{-2}$ at $\nu =$ 1 THz
  (terahertz). Assume the dust perfectly absorbs across its
  cross-section. Ignoring thermal radiation by the dust, calculate
  the profile of $I_\nu$ through the cloud and the optical depth
  through the cloud.
\item Add in the thermal reemission. Assume each dust grain radiates
  as a blackbody  with T$=$ 50 K across its geometric
  cross-section. Calculate $j_\nu$ at 1 THz. Find the functional form
  for and sketch---for the case of {\it no backlight}---the 
  profile $I_\nu$ through the cloud and the calculate the emergent
  $I_\nu$. Include both emission and self-absorption!
\item Now include the backlight and repeat the previous step.
\end{enumerate}

\end{document}

