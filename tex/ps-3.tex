\documentclass[11pt, preprint]{article}
\usepackage{hyperref}
\usepackage{amsmath}
\usepackage{rotating}
 
\setlength{\footnotesep}{9.6pt}
\setlength{\parskip}{12pt}

\newcounter{thefigs}
\newcommand{\fignum}{\arabic{thefigs}}

\newcounter{thetabs}
\newcommand{\tabnum}{\arabic{thetabs}}

\newcounter{address}

\begin{document}

\begin{center}
  {\bf Radiative Processes in Astrophysics / Problem Set \#3 \\
    Due March 3, 2021}
\end{center}

Choose {\bf ONE} of the two problems below to solve.

\begin{enumerate}
\item The effect of a polarization filter or a quarter-wave plate, or
  other device affecting the polarization of light is often expressed
  in terms of a 4-by-4 ``Mueller matrix'' ${\bf M}$ defined such that:
  \begin{equation}
    \begin{pmatrix}
      I_{\rm out} \\
      Q_{\rm out} \\
      U_{\rm out} \\
      V_{\rm out} \\
    \end{pmatrix}
    = 
    {\bf M} 
    \cdot
    \begin{pmatrix}
      I_{\rm in} \\
      Q_{\rm in} \\
      U_{\rm in} \\
      V_{\rm in} \\
    \end{pmatrix}
  \end{equation}

  \begin{enumerate}
    \item Imagine a polarization filter that admits only linearly polarized
  light along an axis, taken to be $\theta$ with respect to the
  $x$-axis. What is the Mueller matrix for this filter?

  \item Now imagine you use this filter on the observations of some
    object and use a CCD to measure the intensity $I_{\rm
      out}$. Assuming $V=0$ (almost always true in astrophysical
    applications), what is the minimum number of measurements with
    different choices of $\theta$ that you have to do to measure the
    polarization fraction $\Pi$?

  \item Most (all?) polarimetry observations use more than that
    minimum number and much cleverer techniques to observe an object
    through various polarization filter simultaneously. Comment on why
    that is a good idea (beyond any extra total signal-to-noise you
    get from more observations).
  \end{enumerate}

\item Consider an interstellar medium with a constant dust density, so
  that the absorption and scattering factors $\alpha_\nu$ and
  $\sigma_\nu$ are constant. Ignore thermal radiation from the
  dust. Assume the scattering is isotropic and coherent.
  \begin{enumerate}
    \item If from the Earth (i.e. from within this interstellar
      medium) I observe the spectrum of a star at some distance $d$
      (using a narrow aperture including just the light coming from
      exactly the direction of the star). By what factor is the flux I
      measure at frequency $\nu$ affected by the intervening dust?
    \item Now imagine I observe the total spectrum of a distant galaxy
      from outside, using an aperture that covers all of the light
      coming from the galaxy. Assume the galaxy is spherical and the
      dust, the stars are intermixed randomly, and for simplicity they
      are all the same type of star as in part (a).  If $\alpha_\nu =
      0$ but $\sigma_\nu$ is significant, how is the spectrum affected
      by the presence of the dust?
    \item If $\sigma_\nu = 0$ but $\alpha_\nu$ is significant, how
      does the effect of dust on the spectrum differ from the effect
      on the single star in part (a)?
    \item If both $\sigma_\nu$ and $\alpha_\nu$ are significant, how
      do you expect the spectrum of the galaxy to differ qualitatively
      from the spectrum of the individual star?
  \end{enumerate}
  
\end{enumerate}

\end{document}

