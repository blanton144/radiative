\documentclass[11pt, preprint]{article}
\usepackage{hyperref}
\usepackage{rotating}
\usepackage[normalem]{ulem}
\usepackage{environ}
\usepackage{xcolor}
\usepackage{amsmath}

 
\setlength{\footnotesep}{9.6pt}
\setlength{\parskip}{12pt}

\newcounter{thefigs}
\newcommand{\fignum}{\arabic{thefigs}}

\newcounter{thetabs}
\newcommand{\tabnum}{\arabic{thetabs}}

\newcounter{address}

\NewEnviron{answer}[1][]{\color{blue}\expandafter\BODY
{\par \it #1}}

\begin{document}

\begin{center}
  {\bf Radiative Processes in Astrophysics / Problem Set \#5 /
    Answers}
\end{center}

\begin{enumerate}
\item 
  The Rosseland mean absorption is the applicable mean absorption
  throughout most of a star:
  \begin{equation}
    \alpha_R^{\rm ff} = \frac{\int {\rm d}\nu \partial B_\nu /
      \partial T}
    {\int {\rm d}\nu \left(1/\alpha_\nu^{\rm ff}\right) \partial B_\nu / \partial T}
  \end{equation}
  Use the formula for free-free absorption:
  \begin{equation}
    \alpha_\nu^{\rm ff} \propto n_e n_i \frac{1}{\nu^3} \left[1 -
      \exp(-h\nu/ kT)\right] T^{-1/2}
  \end{equation}
  to derive the dependence of $\alpha_R^{\rm ff}$ on temperature. You
  do not have to derive the coefficient of the dependence. Note there
  is an annoying integral you will encounter, whose evaluation you do
  not have to do explicitly. This dependence is known as {\it Kramer's
    Opacity Law} and is relevant in low-to-medium mass stars.

  \begin{answer}
    The scaling with $T$ of the numerator of the formula for $\alpha_R^{\rm
      ff}$ can be derived as follows using the Stefan-Boltzmann law:
    \begin{equation}
      \int {\rm d}\nu \frac{\partial B}{\partial T} =
      \frac{\partial}{\partial T} \int {\rm d}\nu B \propto  
      \frac{\partial}{\partial T} T^4 \propto  T^3
    \end{equation}

    The denominator can be simplified as follows:
    \begin{eqnarray}
      \int {\rm d}\nu \left(\frac{1}{\alpha_\nu^{\rm ff}}\right)
      \frac{\partial B}{\partial T}
      &\propto&
      \int {\rm d}\nu T^{1/2} \nu^3 \left(1-
      \exp(-h\nu/kT)\right)^{-1}
      \frac{\partial}{\partial T}\left(\frac{2 h \nu^3
        c^2}{\exp(h\nu/kT) -1}\right) \cr 
     &\propto&
     \int {\rm d}\nu T^{1/2} \nu^6 \left(1- \exp(-h\nu/kT)\right)^{-1}
     \left(\frac{(h\nu/kT^2) \exp(h\nu/kT)
     }{\left(\exp(h\nu/kT) -1\right)^2}\right) \cr 
     &\propto&
     \int {\rm d}\nu T^{-3/2} \nu^7 f\left(\frac{h\nu}{kT}\right)
    \end{eqnarray}
    where we encapsulate the dependence on $h\nu/kT$ in the function
    $f$, and then we can perform the substitution $x = h\nu/kT$:
    \begin{equation}
      \int {\rm d}\nu \left(\frac{1}{\alpha_\nu^{\rm ff}}\right)
      \frac{\partial B}{\partial T} \propto T^{-3/2} T^{8} \int {\rm
        d}x x^7 f(x) \propto T^{13/2}
    \end{equation}
    Then (reinserting the dependence on number densities) the
    Rosseland mean is:
    \begin{equation}
      \alpha_{\rm R}^{\rm ff} \propto n_e n_i T^{-7/2}
    \end{equation}
    Kramer's Law is usually expressed in terms of the opacity $\kappa
    = \alpha/\rho \propto \rho T^{-7/2}$.
  \end{answer}

\item Consider a massive cluster of galaxies, with mass $M\sim
  10^{15}$ $M_\odot$ (total) and radius $R\sim 1$ Mpc. Assume
  the cluster has the cosmic baryon fraction ($\sim 15\%$) and that
  most of the gas is distributed in a large spherical region the size
  of the cluster.
\begin{enumerate}
  \item Estimate the number density of protons and electrons. 

    \begin{answer}
      In terms of the total mass, radius, and the cosmic baryon
      fraction $\Omega_b/\Omega_m \approx 0.15$ we have the number
      density:
      \begin{equation}
        n = \frac{N}{V} = \frac{M (\Omega_b/\Omega_m) / m_p}{4\pi
          R^3/3} = \frac{2\times 10^{71}}{10^{74} {\rm ~cm}^3} \sim
        2\times 10^{-3} {\rm ~cm}^{-3}
      \end{equation}
    \end{answer}
  \item Use the virial theorem relating kinetic energy $K$ and
    potential energy $U$ ($U = -2K$) to find an order-of-magnitude
    temperature of the gas (go ahead an assume for this estimate that
    it is all hydrogen), and the energy in keV of photons at the
    exponential cutoff of free-free emission.

    \begin{answer}
      The potential energy associated with the gas will be of order:
      \begin{equation}
        U \sim - \frac{G M M_b}{R} \sim 
        - \frac{\Omega_b}{\Omega_m} \frac{G M^2}{R} \sim - 5\times 10^{63}
             {\rm erg}
      \end{equation}
      The total kinetic energy in the gas will be of order:
      \begin{equation}
        K = N \frac{3}{2} k T 
      \end{equation}
      and so:
      \begin{equation}
        T \sim -\frac{3 U}{N k} \sim 10^{9} {\rm ~K}
      \end{equation}
      Converting to $h\nu \sim kT$, we find $h\nu\sim 100$ keV. 
      The temperature is an overestimate by about a factor of ten
      relative to a typical rich cluster. 
    \end{answer}
  \item Justify the assumption that the gas is fully ionized.

    \begin{answer}
      If collisions between free electrons and neutral atoms occur at
      all, the energies involved will be far higher than the 13.6 eV
      required to ionize the hydrogen. So virtually every such
      collision will result in an ionization.

      The rate of collisions will be reasonably high. The typical
      velocity of a hydrogen atom would be:
      \begin{equation}
        v \sim \sqrt{\frac{kT}{m_p}} \sim 1,000 {\rm ~km} {\rm
          ~s}^{-1}
      \end{equation}
      and the cross-section is something like $\sigma = \pi a_0^2$,
      where $a_0 = 5.3\times 10^{-9} {\rm ~cm}$ is the Bohr radius.
      The rate of collisions is $n \sigma v \sim 2 \times 10^{-11}$
      s$^{-1}$, or once every 1,000 years or so. So there is ample
      opportunity to ionize. The recombination rate will be very much
      lower, since most proton-electron encounters will be at energies
      far higher than 13.6 eV.
    \end{answer}
  \item Compare the optical depth to Thomson scattering in this system
    to the optical depth to free-free absorption for photons with
    $h\nu \sim kT$. What is the total effective optical depth? Recall
    the results we discussed earlier in the semester regarding optical
    depth in the case of absorption and scattering! Is the system
    optically thin?

    \begin{answer}
      The mean free path for Thomson scattering is:
      \begin{equation}
        l = \frac{1}{n \sigma_T} \sim 10^{27} {\rm ~cm} \sim 300 {\rm
          ~Mpc}
      \end{equation}
      and thus the optical depth $\tau_s \sim R/l \sim 0.003$, 
      indicating that it is very very optically thin to scattering.

      Using the formula in R\&L for the free-free absorption in cgs
      units:
      \begin{equation}
        \alpha_\nu^{\rm ff} = (3.7\times 10^8) T^{-1/2} Z^2 n_e n_i
        \nu^{-3} \left(1 - e^{-h\nu/kT}\right) {\bar g}_{\rm ff}
      \end{equation}
      we find for $Z=1$, ${\bar g}_{\rm ff} = 1$, and $h\nu=kT$ that
      $\alpha_\nu \sim 10^{-60}$ cm$^{-1}$, which means the mean free
      path $l \sim 10^{60}$ cm $\sim 10^{35}$ Mpc, i.e. extremely
      large, leading to a very close to zero optical depth to
      absorption ($\tau_a \sim 10^{-35}$).

      Clearly the cluster is extremely optically thin to its own
      thermal X-rays. But for completeness the ``effective'' optical
      depth is:
      \begin{equation}
        \tau_\ast \approx \sqrt{\tau_a(\tau_a + \tau_s)} \sim
      \sqrt{\tau_a \tau_s} \sim 10^{-16}.
       \end{equation}
    \end{answer}
  \item If the metallicity of the gas is roughly solar (0.02 by mass),
    how important is the contribution of higher mass ions likely to be
    to the level of free-free emission? Give just an order of
    magnitude estimate (and just assume all the metals are oxygen for
    simplicity).

    \begin{answer}
      The emission scales as follows (assuming the Gaunt factor
      differences are negligible):
      \begin{equation}
        j_\nu \propto Z^2 n_e n_i
      \end{equation}
      Assuming all the metals are ${}^{16}O$, the factor $Z^2$ will
      increase the emission by a factor 64. The number density of
      oxygen ions will be about 0.001 the number density of
      protons. There will be up to 8 time more free electrons per
      ion than for ions, but that is still only a fractional
      difference of about 0.01. So overall the increase in $j_\nu$
      will be 5--10\%, or fairly modest.
      \end{answer}
\end{enumerate}

\end{enumerate}


\end{document}

