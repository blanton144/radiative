\documentclass[11pt, preprint]{article}
\usepackage{hyperref}
\usepackage{rotating}
\usepackage[normalem]{ulem}
\usepackage{environ}
\usepackage{xcolor}
\usepackage{amsmath}

 
\setlength{\footnotesep}{9.6pt}
\setlength{\parskip}{12pt}

\newcounter{thefigs}
\newcommand{\fignum}{\arabic{thefigs}}

\newcounter{thetabs}
\newcommand{\tabnum}{\arabic{thetabs}}

\newcounter{address}

\NewEnviron{answer}[1][]{\color{blue}\expandafter\BODY
{\par \it #1}}

\begin{document}

\begin{center}
  {\bf Radiative Processes in Astrophysics / Problem Set \#5 /
    Answers}
\end{center}

\begin{enumerate}
\item 
  The Rosseland mean absorption is the applicable mean absorption
  throughout most of a star:
  \begin{equation}
    \alpha_R^{\rm ff} = \frac{\int {\rm d}\nu \partial B_\nu /
      \partial T}
    {\int {\rm d}\nu \left(1/\alpha_\nu^{\rm ff}\right) \partial B_\nu / \partial T}
  \end{equation}
  Use the formula for free-free absorption:
  \begin{equation}
    \alpha_\nu^{\rm ff} \propto n_e n_i \frac{1}{\nu^3} \left[1 -
      \exp(-h\nu/ kT)\right] T^{-1/2}
  \end{equation}
  to derive the dependence of $\alpha_R^{\rm ff}$ on temperature. You
  do not have to derive the coefficient of the dependence. Note there
  is an annoying integral you will encounter, whose evaluation you do
  not have to do explicitly. This dependence is known as {\it Kramer's
    Opacity Law} and is relevant in low-to-medium mass stars.

  \begin{answer}
    you 
  \end{answer}

\item Consider a massive cluster of galaxies, with mass $M\sim
  10^{15}$ $M_\odot$ (total) and radius $R\sim 1$ Mpc. Assume
  the cluster has the cosmic baryon fraction ($\sim 15\%$) and that
  most of the gas is distributed in a large spherical region the size
  of the cluster.
\begin{enumerate}
  \item Estimate the number density of protons and electrons. 
  \item Use the virial theorem relating kinetic energy $K$ and
    potential energy $U$ ($U = -2K$) to find an order-of-magnitude
    temperature of the gas (go ahead an assume for this estimate that
    it is all hydrogen), and the energy in keV of photons at the
    exponential cutoff of free-free emission.
  \item Justify the assumption that the gas is fully ionized.
  \item Compare the optical depth to Thomson scattering in this system
    to the optical depth to free-free absorption for photons with
    $h\nu \sim kT$. What is the total effective optical depth? Recall
    the results we discussed earlier in the semester regarding optical
    depth in the case of absorption and scattering! Is the system
    optically thin?
  \item If the metallicity of the gas is roughly solar (0.02 by mass),
    how important is the contribution of higher mass ions likely to be
    to the level of free-free emission? Give just an order of
    magnitude estimate (and just assume all the metals are oxygen for
    simplicity).
\end{enumerate}

\end{enumerate}


\end{document}

