\documentclass[11pt, preprint]{article}
\usepackage{hyperref}
\usepackage{amsmath}
\usepackage{rotating}
 
\setlength{\footnotesep}{9.6pt}
\setlength{\parskip}{12pt}

\newcounter{thefigs}
\newcommand{\fignum}{\arabic{thefigs}}

\newcounter{thetabs}
\newcommand{\tabnum}{\arabic{thetabs}}

\newcounter{address}

\begin{document}

\begin{center}
  {\bf Radiative Processes in Astrophysics / Problem Set \#4 \\
    Due March 10, 2021}
\end{center}

\begin{enumerate}
\item The cosmic microwave background at $z\sim 1100$ is the last
  scattering surface of the photons from the hot ionized gas that
  filled the universe before that time. Before recombination, these
  photons Thomson scatter efficiently off the electons in the ionized
  gas, and are kept in thermal equilibrium at about $T\sim 3000$ K at
  that epoch. As hydrogen atoms recombine at $z\sim 1100$, the gas
  becomes transparent to most of these photons, which then travel
  towards us.
\begin{enumerate}
\item Using the known cosmic baryon density, estimate the mean free
  path (in physical units) of a photon to Thomson scattering when the
  ionization fraction is 0.5.
\item The photons reaching us are those emitted exactly normal to the
  surface defined by the recombination epoch. These photons are the
  result of scattering from the gas surrounding the point in
  question. If the temperature of the CMB were uniform, do you expect
  the light reaching us to be polarized, and if why or why not?
\item Considering a local patch of the reionization surface, sketch a
  pattern of temperature fluctuations on the surface that would yield
  a net polarization.
\end{enumerate}

\item Consider the dependence of the ``equivalent width'' associated
  with an absorption line with a Voigt profile, on the optical depth
  at line center. Assumed the absorbed continuum is flat in
  $f_\lambda$. You may treat ${\rm d}\lambda/\lambda \propto {\rm
    d}\nu/\nu$ in the region of the line. The equivalent width is the
  integral of the absorbed light in $f_\lambda$ divided by the flux
  density of the continuum in $f_\lambda$.
  \begin{enumerate}
    \item For $\tau\ll 1$, approximate the dependence of equivalent
      width on $\tau$. 
    \item For $\tau\gg 1$, neglecting the Lorentzian term
      (i.e. $\Gamma = 0$), approximate the dependence of equivalent
      width on $\tau$.
    \item For $\tau\gg 1$, neglecting the Doppler term
      (i.e. $\sigma = 0$), approximate the dependence of equivalent
      width on $\tau$.
    \item Assume $\Lambda$ is the classical damping width, and the
      velocity dispersion $\sigma =$ 10 km s$^{-1}$. Based on the
      scalings you just calculated, sketch (the log of) equivalent
      width vs $\tau$.
  \end{enumerate}
  
\end{enumerate}

\end{document}

