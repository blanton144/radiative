\documentclass[11pt, preprint]{article}
\usepackage{hyperref}
\usepackage{rotating}
\usepackage[normalem]{ulem}
\usepackage{environ}
\usepackage{xcolor}

 
\setlength{\footnotesep}{9.6pt}
\setlength{\parskip}{12pt}

\newcounter{thefigs}
\newcommand{\fignum}{\arabic{thefigs}}

\newcounter{thetabs}
\newcommand{\tabnum}{\arabic{thetabs}}

\newcounter{address}

\NewEnviron{answer}[1][]{\color{blue}\expandafter\BODY
{\par \it #1}}

\begin{document}

\begin{center}
  {\bf Radiative Processes in Astrophysics / Problem Set \#1 /
    Answers}
\end{center}

\begin{enumerate}
\item Show that for an optically thin cloud around a source with mass
  $M$ and luminosity $L$ (integrated over frequency), the condition
  that the luminosity drives away the cloud through radiation pressure
  is $L>4\pi G M c/\kappa$, where $\kappa$ is the opacity
  (i.e. absorption coefficient per unit mass, assumed constant with
  frequency).

\begin{answer}
Assume that it is a point source of mass and luminosity. Consider a
small volume of material at distance $r$. The force per unit mass
(i.e. the acceleration) due to gravity is:
\begin{equation}
a_g = - \frac{GM}{r^2}.
\end{equation}
Meanwhile, from R\&L Equation 1.34 we know that the force per unit
mass due to radiation is:
\begin{equation}
f_r = \frac{1}{c} \int \rm{d}\,\nu F_\nu \kappa_\nu
\end{equation}
Let's take $\kappa$ to be the flux-weighted average of
$\kappa_\nu$. Then:
\begin{equation}
f_r = \frac{\kappa}{c} F = \frac{\kappa L}{4\pi r^2 c}
\end{equation}
Then the condition that $f_r$ overpowers $f_g$ is:
\begin{eqnarray}
f_r &>& f_g \cr
\frac{\kappa L}{4 \pi r^2 c} &>& \frac{GM}{r^2} \cr
L &>& \frac{4 \pi GM c}{\kappa}.
\end{eqnarray}
This result is independent of radius.

Pretty much the minimum $\kappa$ will be that due to Thomson
scattering, and when Thomson scattering is used then this luminosity
limit is known as the Eddington luminosity or Eddington limit. The
Eddington limit seems to be respected for the most part by accreting
black holes, even though they are definitely not spherically accreting
and are not optically thin.
\end{answer}
\item Sunspots on the Sun have a temperature of $\sim 4000$ K,
  relative to the typical location on the Sun, which has a temperature
  of $\sim 5500 K$. What is the ratio of the specific intensity
  integrated over frequency ($\int d\nu I_\nu$) that we should observe
  in the location of a Sunspot relative to a typical location on the
  solar disk?

\begin{answer}
The bolometric intensity will scale as $T^4$. Therefore the ratio of
that observed in the sunspot vs. the other locations on the Sun is
$(4000/5500)^4 \sim 0.28$. When you observe sunspots (through a
specially designed instrument with a very strong solar filter! do not
try to observe the Sun in any other way!!) they thus appear to your
eye very dark even though they are only 30\% cooler. This is due to
the strong dependence of the intensity on temperature.
\end{answer}
\item Light that reaches us from the Sun's ``limb'' (the area near its
  apparent angular edge) has a lower specific intensity than light
  that reaches us from its angular center. Explain why. Do you predict
  that the spectrum as a function of frequency differs in shape from
  center to edge, and if so how?

\begin{answer}
Consider the volume of the Sun near its surface, which is within an
optical depth of unity of its surface. This region is known as the
photosphere. There is a significant temperature gradient within the
photosphere, declining with radius. 

When observing the center of the solar disk, the radial depth into the
Sun corresponding to $\tau \sim 1$ is maximal, and thus the
temperature of the emitting material at $\tau\sim 1$ is highest and
the specific intensity is the largest.

When observing the edge of the solar disk, because the ray is not
normal to the radial direction in the Sun, the radial depth into the
Sun corresponding to $\tau\sim 1$ is smaller, and the temperature  of
the emitting material at $\tau\sim 1$ is lower.

As we saw in the previous problem, there is a strong dependence of the
intensity on temperature, and so the edge of the solar disk will
appear substantially darker in its specific intensity, an effect
called ``limb darkening.'' This effect is observed in planetary
transits around distance stars as well, seen in the shape of the
transit light curves.

We expect the spectrum to differ as well, with the limb being cooler
(redder) emission; there will be other effects on absorption line
shapes and depths that will be a bit subtler.
\end{answer}

\item Given that for redshifted light $v_o = \nu_e / (1+z)$ show that
  $I_\nu \nu^{-3}$ is a constant. 

\begin{answer}
  First, we remind ourselves that the definition of specific intensity is:
  \begin{equation}
    I_\nu = \frac{{\rm d}E}{{\rm d}\nu {\rm d}A {\rm d}t {\rm
        d}\Omega}
  \end{equation}
  Second, we consider the number of photons in some small volume of
  phase space, in terms of the distribution function $f$:
  \begin{equation}
    {\rm d}N = f(\vec{x}, \vec{p}) {\rm d}^3 \vec{x} {\rm d}^3 \vec{p}
  \end{equation}
  The energy in that volume associated with photons of frequency $\nu
  = p c/h$ corresponding to momentum $p$ can be obtained by
  multiplying by $h\nu$, and can be rearranged as follows:
  \begin{eqnarray}
    {\rm d}E &=&
    h \nu f(\vec{x}, \vec{p}) {\rm d}^3 \vec{x} {\rm d}^3 \vec{p} \cr
    &=&
    h \nu f(\vec{x}, \vec{p}) \left(c {\rm d}t {\rm d}A\right)
     \left[ \left(\frac{h}{c}\right)^3 \nu^2 {\rm d}\Omega {\rm
         d}\nu\right],
  \end{eqnarray}
  where the spatial volume element is re-expressed as the volume
  traversed by photons traveling during time ${\rm d}t$ in direction
  $\hat{p}$ in a cross-sectional area ${\rm d}A$, and the momentum
  volume element is reexpressed in terms of the frequency using $p =
  E/c = h\nu /c$ and in terms of the differential frequency ${\rm
    d}\nu$ and the range of directions (solid angle) ${\rm d}\Omega$.
  This version of the expression allows us to construct $I_\nu$:
  \begin{equation}
    I_\nu = \frac{{\rm d}E}{{\rm d}\nu {\rm d}A {\rm d}t {\rm
        d}\Omega} = \frac{h^4 \nu^3}{c^2} f(\vec{x}, \vec{p})
  \end{equation}
  Liouville's theorem (which holds in general relativity as well as in
  other contexts) says that under collisionless conditions, if
  particles cannot be created or destroyed, then along the path of a
  particle $f(\vec{x}, \vec{v})$ remains a constant. Therefore $I_\nu
  \nu^{-3}$ is a constant.

  Given the wording of the question, it seems worth pointing out that
  in a cosmological context this means that:
  \begin{equation}
    I_{\nu, {\rm obs}} = I_{\nu, {\rm emitted}} \frac{\nu_{\rm
        obs}^3}{\nu_{\rm emitted}^3} = I_{\nu, {\rm emitted}} (1+z)^{-3}
  \end{equation}
  and if we integrate over frequency:
  \begin{eqnarray}
    I_{\rm obs}
     &=& \int {\rm d}\nu_{\rm obs} I_{\nu, {\rm obs}}(\nu_{\rm obs}) \cr
 &=& \int {\rm d}\nu_{\rm emitted} (1+z)^{-1} I_{\nu, {\rm
     emitted}}(\nu_{\rm emitted}) (1+z)^{-3} \cr
     &=& (1+z)^{-4} \int {\rm d}\nu_{\rm emitted} I_{\nu, {\rm
        emitted}} \cr
 &=& (1+z)^{-4} I_{\rm emitted}.
  \end{eqnarray}
This effect is generally known within cosmology as ``$(1+z)^4$
surface brightness dimming'' and is a very important effect in
observing the distant universe.
\end{answer}
  
\item In cosmology the angular diameter distance is defined as $D_A =
  s / \theta$, where $s$ is the physical size of an object and
  $\theta$ is its angular size, and the luminosity distance is defined
  as $D_L = \sqrt{L/4\pi f}$, where $L$ is the total luminosity of an
  object (integrated over all frequencies) and $f$ is its measured
  flux. Using only the fact that $I_\nu \nu^{-3}$ is a constant show
  that $D_L/D_A = (1+z)^2$.

  \begin{answer}
    Given the answer to the previous question, we start by noting that
    the bolometric intensity $I_{\rm obs} \propto (1+z)^{-4}$ (and has no
    other dependence on distance.

    The flux of an object can be written in terms of the bolometric
    intensity and the solid angle it subtends:
    \begin{equation}
     f = I \Omega
    \end{equation}

    The definition of $D_A$ tells us that:
    \begin{equation}
      \Omega \propto \frac{1}{D_A^2}
    \end{equation}
    with no other dependence on distance or redshift.

    The definition of $D_L$ tells us that:
    \begin{equation}
      f \propto \frac{1}{D_L^2}
    \end{equation}
    with no other dependence on distance or redshift.

    Therefore:
    \begin{eqnarray}
      f &\propto& \frac{1}{D_L^2} \quad{\rm ~and~is~also}\cr
      &=& I \Omega \propto \frac{1}{(1+z)^4} \frac{1}{D_A^2}.
    \end{eqnarray}

    Jointly, this pair of proportionalities requires:
    \begin{equation}
      \frac{D_L}{D_A} = (1+z)^2
    \end{equation}

    That is, the relationship between the luminosity and angular
    diameter distance is {\it purely} a statement of Liouville's
    theorem. Any violation of this relationship implies that the
    conditions for Liouville's theorem to hold are being violated:
    photons are being created or destroyed, for example. This
    relationship can be (has been in fact) tested by using standard
    rulers (BAO) and standard candles (SNIa) to measure $D_A$ and
    $D_L$ independently. 
  \end{answer}
\end{enumerate}

\end{document}

