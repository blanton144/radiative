\documentclass[11pt, preprint]{article}
\usepackage{hyperref}
\usepackage{amsmath}
\usepackage{rotating}
 
\setlength{\footnotesep}{9.6pt}
\setlength{\parskip}{12pt}

\newcounter{thefigs}
\newcommand{\fignum}{\arabic{thefigs}}

\newcounter{thetabs}
\newcommand{\tabnum}{\arabic{thetabs}}

\newcounter{address}

\begin{document}

\begin{center}
  {\bf Radiative Processes in Astrophysics / Problem Set \#8 \\
    Due May 3, 2021}
\end{center}

\begin{enumerate}
\item In Shengqi Yang's PhD defense last week she discussed the IR
  lines within the ground configuration of OIII, which is
  1s$^2$2s$^2$2p$^2$. For the IR transitions within the ${}^3$P term,
  she considered, and the plots of the OIII transitions within this
  configuration always show, just the 52 $\mu$m line (between $J=2$
  and $J=1$) and the 88 $\mu$m line (between $J=1$ and $J=0$). But
  there also is a potential transition between $J=2$ and
  $J=0$. Determine what type of transition that third one is,
  i.e. electric dipole, electric quadrupole, or magnetic dipole, and
  explain why in terms of the selection rules. Show how the spacing
  between the three transitions results from Land\'e's interval rule
  for spin-orbit interactions. If you want to see why the third line
  is usually omitted you can look it up using the line lists on NIST,
  and you will see why Shengqi is safe in ignoring the transition with
  current technology.
\item Also in the same presentation by Dr.~Yang (as she is now!) she
  discussed the variation of the line ratio between 52 $\mu$m and 88
  $\mu$m as a function of electron density, showing a transition
  occurring somewhere in the range $n_e\sim10^2$--$10^3$ cm$^{-3}$. It
  is this variation in the line ratio that allows a constraint on $n_e$.
\begin{enumerate}
  \item Explain qualitatively what is going on---i.e. why is there a
    special density at which this line ratio is likely to change? How
    will that density depend on the $q$ and $A$ values among the three
    levels?
  \item Dr.~Wang presented constraints on metallicity and electron
    density; but she did not talk about the temperature of the gas
    (i.e. the electron temperature).  Argue why the relative
    populations of the levels of the ${}^3$P term will not depend on
    temperature, justifying why she didn't talk about it.  (Remember
    OIII only exists in gas that is ionized, either collisionally or
    photoionized).
  \item Write the equations for the balance between the three energy
    states, in terms of the number densities $n_0$, $n_1$, and $n_2$,
    ad in terms of $q_{ij}$ and $A_{ij}$.  You should end up with a
    homogeneous linear system of three equations. I'm not asking you
    to solve the system fully (though it can be done). Also, you can
    leave in factors like $q_{02}$ and $q_{20}$; i.e. you don't need
    to use the relationship between those two rates imposed by
    detailed balance considerations.
  \item Instead, let's think about the low density limit, when
    $n_e\rightarrow 0$. Use the equations for $n_2$ and $n_1$ in order
    to find two equations, one for $n_2/n_0$ and one for $n_1/n_0$, in
    terms of each other. Use the assumption that $n_e$ is small to
    find approximations for $n_2/n_0$ and $n_1/n_0$ to first order in
    $n_e$ (it should be a very simple formula for each!). Finally,
    determine what the relative line flux will be between the $J=2$ to
    $J=1$ vs. the $J=1$ ro $J=0$ transition under these conditions.
  \item In the high density limit, we can assume the levels will be
    kept in thermodynamic equilibrium with the electrons. Determine
    the relative line flux in this case (remember part (b), note that
    the multiplicities of each state will come into the calculation,
    and also leave things in terms of $q$ and $A$ when necessary!).
\end{enumerate}
\end{enumerate}

\end{document}

