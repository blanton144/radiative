\documentclass[11pt, preprint]{article}
\usepackage{hyperref}
\usepackage{amsmath}
\usepackage{rotating}
 
\setlength{\footnotesep}{9.6pt}
\setlength{\parskip}{12pt}

\newcounter{thefigs}
\newcommand{\fignum}{\arabic{thefigs}}

\newcounter{thetabs}
\newcommand{\tabnum}{\arabic{thetabs}}

\newcounter{address}

\begin{document}

\begin{center}
  {\bf Radiative Processes in Astrophysics / Problem Set \#7 \\
    Due April 26, 2021}
\end{center}

There are a few questions here but they are mostly exercises to
familiarize you with the typical scales involved regarding the various
effects we have discussed recently.

\begin{enumerate}
\item Compton scattering of CMB photons of electrons in hot gas in
  galaxy clusters leads to a distortion of the spectrum. The
  distortion is quantifiable by the Compton parameter $y=N kT_e /
  m_ec^2$, where $N$ is the number of scatterings that occur.
  \begin{enumerate}
    \item Explain in words (or maybe a picture) why the spectral
      distortion can be distinguished from variation of the background
      CMB temperature from measuring the spectrum alone.
    \item Estimate $y$ for a galaxy cluster (use the total optical
      depth to scattering of $\tau\sim 0.003$ from Problem Set \#5,
      and assume $T_e\sim 10^8$ K). On the Rayleigh-Jeans tail of the
      CMB light, what is the induced fractional difference in the
      intensity of observed light? 
  \end{enumerate}
\item Imagine a set of Cherenkov detectors on the wall of a big vat of
  water, say with refractive index of $1.3$. If an electron is created
  that is heading directly at the wall and briefly emits Cherenkov
  radiation, it will lead to a set of detections arranged in a
  ring. Write down how the size of the ring depends on distance to the
  wall and $\beta$ for the electron. 
\item For a Galactic magnetic field of $B\sim \mu$G, and assuming a
  mean electron density of $n_e \sim 0.1$ cm$^{-3}$, what will ${\rm
    d}t/{\rm d}\nu$ (i.e. dispersion) and ${\rm d}\theta/{\rm d}\nu$
  (i.e. Faraday rotation) be of a pulsar 1 kpc away be, assuming the
  magnetic field is toward us along the line of sight. If the magnetic
  field in the Galactic disk were oriented in azimuthally symmetric
  circles (it is not!) how would the rotation measure depend on the
  observed Galactic longitude of the pulsar?
\item A calcium atom is in configuration
  1s$^2$2s$^2$2p$^6$3s$^2$3p$^6$3d5p. What terms are associated with
  this configuration and which do you expect to be lowest in energy
  based on Hund's rules? What levels can arise from the lowest energy
  term?
\end{enumerate}

\end{document}

