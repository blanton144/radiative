\documentclass[11pt, preprint]{article}
\usepackage{hyperref}
\usepackage{rotating}
\usepackage[normalem]{ulem}
\usepackage{environ}
\usepackage{xcolor}
\usepackage{amsmath}

 
\setlength{\footnotesep}{9.6pt}
\setlength{\parskip}{12pt}

\newcounter{thefigs}
\newcommand{\fignum}{\arabic{thefigs}}

\newcounter{thetabs}
\newcommand{\tabnum}{\arabic{thetabs}}

\newcounter{address}

\NewEnviron{answer}[1][]{\color{blue}\expandafter\BODY
{\par \it #1}}

\begin{document}

\begin{center}
  {\bf Radiative Processes in Astrophysics / Problem Set \#7 /
    Answers}
\end{center}

\begin{enumerate}
\item Compton scattering of CMB photons of electrons in hot gas in
  galaxy clusters leads to a distortion of the spectrum. The
  distortion is quantifiable by the Compton parameter $y=N kT_e /
  m_ec^2$, where $N$ is the number of scatterings that occur.
  \begin{enumerate}
    \item Explain in words (or maybe a picture) why the spectral
      distortion can be distinguished from variation of the background
      CMB temperature from measuring the spectrum alone.

      \begin{answer}
        Temperature fluctuations lead to a change in specific
        intensity which is monotonic with temperature at all
        wavelengths. Compton scattering does not lead the radiation
        temperature to change but applies a spectral distortion that
        redistributes energy; the low energy spectrum actually
        decreases and the high energy spectrum (roughly above the
        spectral peak) increases. The resulting spectrum is not 
        a Planck spectrum of any temperature.
      \end{answer}
    \item Estimate $y$ for a galaxy cluster (use the total optical
      depth to scattering of $\tau\sim 0.003$ from Problem Set \#5,
      and assume $T_e\sim 10^8$ K). On the Rayleigh-Jeans tail of the
      CMB light, what is the induced fractional difference in the
      intensity of observed light? 

      \begin{answer}
        We can use the definition of $y$ and all we really need to
        know is the number of scatterings, which at low optical depth
        is just $\tau$. Then we can convert $kT_e$ to keV to compare
        to $m_ec^2 = 511$ keV:
        \begin{equation}
          y  = \frac{N kT_e}{m_e c^2} = \frac{\tau kT_e}{m_e c^2} =
          \frac{(0.003) (8.6 {\rm ~keV})}{511 {\rm ~keV}} = 5 \times
          10^{-5}
        \end{equation}
        So $y$ is very small!! The fractional change in intensity on
        the Raleigh-Jeans tail is $-2y = -10^{-4}$, so very small
        (though still bigger than the CMB fluctuations).
      \end{answer}
  \end{enumerate}
\item Imagine a set of Cherenkov detectors on the wall of a big vat of
  water, say with refractive index of $1.3$. If an electron is created
  that is heading directly at the wall and briefly emits Cherenkov
  radiation, it will lead to a set of detections arranged in a
  ring. Write down how the size of the ring depends on distance to the
  wall and $\beta$ for the electron. 

  \begin{answer}
    Relative to the direction of motion, Cherenkov radiation is
    emitted at an angle $\theta$ that satisfies $\cos\theta = 1/\beta
    n_r$. If it emits only for a short time, this results not in a
    full cone but just a ring of light. If this emission is a distance
    $d$ from the wall of detectors, by the time the ring hits the wall
    it will have a size:
    \begin{equation}
      s = d \tan\theta = d \sqrt{\frac{1}{\cos^2\theta} - 1} =
      d\sqrt{\beta^2 n_r^2 -1}
    \end{equation}
    At maximum $\beta\approx 1$ and $n_r\approx 1.3$ this is $s = 0.83
    d$.
  \end{answer}
\item For a Galactic magnetic field of $B\sim \mu$G, and assuming a
  mean electron density of $n_e \sim 0.1$ cm$^{-3}$, what will ${\rm
    d}t/{\rm d}\nu$ (i.e. dispersion) and ${\rm d}\theta/{\rm d}\nu$
  (i.e. Faraday rotation) be of a pulsar 1 kpc away be, assuming the
  magnetic field is toward us along the line of sight. If the magnetic
  field in the Galactic disk were oriented in azimuthally symmetric
  circles (it is not!) how would the rotation measure depend on the
  observed Galactic longitude of the pulsar?

  \begin{answer}
    From the class notes, the change in arrival time as a function of
    frequency is:
    \begin{equation}
      \frac{{\rm d}t}{{\rm d}{\omega}} = - \frac{4\pi e^2}{m_e c}
        \frac{1}{\omega^3} D
    \end{equation}
    where $E$ is the dispersion measure, or:
    \begin{equation}
      \frac{{\rm d}t}{{\rm d}{\nu}} =
        - \frac{e^2}{\pi m_e c} \frac{1}{\nu^3} D  =
        - \frac{e^2}{\pi m_e c} \frac{1}{\nu^3} \int_0^d {\rm d}s n_e =
        - \frac{e^2 n_e d}{\pi m_e c} \frac{1}{\nu^3}
    \end{equation}
    which plugging in numbers is:
    \begin{equation}
      \frac{{\rm d}t}{{\rm d}{\nu}} \sim \left(10^{18} {\rm
        ~s}^{-1}\right) \frac{1}{\nu^3}
    \end{equation}
    At $\nu \sim 1$ GHz this will therefore lead to: 
    \begin{equation}
      \frac{{\rm d}t}{{\rm d}{\ln \nu}} \sim 1 {\rm ~s},
    \end{equation}
    meaning over a broad band pass the change in time delay will be 
    comparable to the period of the pulsar (around one second) and
    therefore should be easily measurable.

    The rotation angle is:
    \begin{equation}
      \theta = \frac{e^3}{2\pi m^2 c^2 \nu^2} \int_0^d {\rm d}s n_e
        B_{||}
    \end{equation}
    and taking the derivative:
    \begin{equation}
      \frac{{\rm d}\theta}{{\rm d}\nu} =
      - \frac{e^3}{\pi m^2 c^2 \nu^3} \int_0^d {\rm d}s n_e
        B_{||} = 
      - \frac{e^3n_e B_{||} d}{\pi m^2 c^2 \nu^3}
    \end{equation}
    Plugging in numbers we find:
    \begin{equation}
      \frac{{\rm d}\theta}{{\rm d}{\nu}} \sim \left(10^{19} {\rm
        ~s}^{-1}\right) \frac{1}{\nu^3}
    \end{equation}
    and again we can check 1 GHz:
    \begin{equation}
      \frac{{\rm d}\theta}{{\rm d}{\ln\nu}} \sim 10
    \end{equation}
    which again indicates that the change in polarization angle will
    be significant across a bandpass.

    The rotation measure depends on the magnetic field projected on
    the line of sight. So if the magnetic field was in circles around
    the Galactic center, the as you looked as a function of Galactic
    longitude it would be zero at $l=0^\circ$ towards the Galactic
    center, increase to a maximum amplitude at $l=90^\circ$, then decrease to
    $l=180^\circ$ at the anticenter, and then reverse sign and
    decrease to a maximum amplitude at $l=270^\circ$.
    
  \end{answer}
\item A calcium atom is in configuration
  1s$^2$2s$^2$2p$^6$3s$^2$3p$^6$3d5p. What terms are associated with
  this configuration and which do you expect to be lowest in energy
  based on Hund's rules? What levels can arise from the lowest energy
  term?

  \begin{answer}
   All of the electrons except for 3d5p are in closed shells, so do
   not contribute to the total spin or orbital angular momentum. With
   two electrons in unclosed shells, $S=0$ or $1$. These electrons
   have $l=2$ and $l=1$, so $L=1$, $2$, or $3$. We can proceed with
   enumerating the terms without being concerned about Pauli
   exclusion, because the $n$ values for the electrons are
   different.

   That means that the available terms associated are ${}^1$P$^o$,
   ${}^1$D$^o$, ${}^1$F$^o$, ${}^3$P$^o$, ${}^3$D$^o$, and ${}^3$F$^o$
   (where $\sum l = 3$ leads to odd parity for all terms). Hund's
   rules says that higher $S$ are lower energy, and within the terms
   of a given $S$ the higher $L$ are lower energy, so the lowest
   energy term will be ${}^3$F$^o$.

   This lowest energy term will have levels associated with $J=2$,
   $3$, and $4$ (from lowest to highest energy), denoted
   ${}^3$F$^o_2$, ${}^3$F$^o_3$, and ${}^3$F$^o_4$.
  \end{answer}
\end{enumerate}

\end{document}

