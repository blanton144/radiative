\documentclass[11pt, preprint]{article}
\usepackage{hyperref}
\usepackage{rotating}
\usepackage[normalem]{ulem}
\usepackage{environ}
\usepackage{xcolor}
\usepackage{amsmath}

 
\setlength{\footnotesep}{9.6pt}
\setlength{\parskip}{12pt}

\newcounter{thefigs}
\newcommand{\fignum}{\arabic{thefigs}}

\newcounter{thetabs}
\newcommand{\tabnum}{\arabic{thetabs}}

\newcounter{address}

\NewEnviron{answer}[1][]{\color{blue}\expandafter\BODY
{\par \it #1}}

\begin{document}

\begin{center}
  {\bf Radiative Processes in Astrophysics / Problem Set \#6 /
    Answers}
\end{center}

\begin{enumerate}
\item Rybicki \& Lightman problem 4.3. Feel free to use the solutions
  to guide your work, but it is worth working this problem through
  fully!

  \begin{enumerate}
    \item Show that the transformation of acceleration under a boost
      of velocity $v$ along $x$ is:
      \begin{eqnarray}
        a_x &=& \frac{a_x'}{\gamma^3 \sigma^3} \cr
        a_y &=& \frac{a_y'}{\gamma^2 \sigma^2} - \frac{u_y' v}{c^2}
        \frac{a_x'}{\gamma^2 \sigma^3}\cr
        a_z &=& \frac{a_z'}{\gamma^2 \sigma^2} - \frac{u_z' v}{c^2}
        \frac{a_x'}{\gamma^2 \sigma^3}
      \end{eqnarray}
      where:
      \begin{equation}
        \sigma = 1 + \frac{v u_x'}{c^2}
      \end{equation}
      and $\vec{u}$ is the three-velocity of the particle in the
      primed frame.
      
      \begin{answer}
        Finding the transformation of acceleration is trickier than it
        at first would appear. A key thing to note is that spatial
        components of the 4-acceleration are {\it only} equal to the
        acceleration as measured in the frame {\it if} one is working
        in the momentarily comoving frame of the particle. So the work
        of finding the acceleration is more laborious than it would be
        if it was easily calculated from the 4-vectors.

        Instead, we have to consider an interval of time in the primed
        frame, during which the particle is moving and its velocity is
        changing in that primed frame.  Then we need to consider how
        that small change in velocity is observed in the unprimed
        frame over the same interval, {\it and} we have to account for
        the difference in the time interval between the two frames.

        So we start with the Lorentz tranformation of the time. During
        the time interval d$t'$ the particle moves d$x'$ and this must
        be accounted for in finding the interval in the unprimed frame.
        \begin{eqnarray}
          {\rm d}t &=& \gamma\left({\rm d}t' + \frac{v}{c^2} {\rm
            d}x'\right) \cr
          &=& \gamma\left(1 + \frac{v}{c^2} \frac{{\rm d}x'}{{\rm d}t}\right)
            {\rm d}t'\cr
          &=& \gamma\left(1 + \frac{v u_x'}{c^2}\right)
            {\rm d}t' = \gamma \sigma {\rm d}t'
        \end{eqnarray}
        where we use the definition of $\sigma$ in the last step.

        Then we consider the Lorentz transformation of the velocity in
        the direction of the boost:
        \begin{equation}
          u_x = \frac{u_x' + v}{1+v u_x'/c^2}
        \end{equation}
        In the interval of time d$t'$ there is a change in velocity
        d$u_x'$ in the primed frame. Let's find what that differential
        turns into in the unprimed frame:
        \begin{eqnarray}
          {\rm d}u_x &=& \frac{{\rm d}u_x'\left(1 + vu_x'/c^2\right) -
            {\rm d}u_x' \left(v/c^2\right)(u_x' +v)}{\left(1+v
            u_x'/c^2\right)^2} \cr
          &=& \frac{{\rm d}u_x'\left(1 - \left(v^2/c^2\right)\right)}
            {\left(1+v u_x'/c^2\right)^2} = \frac{{\rm
                d}u_x'}{\gamma^2 \sigma^2}
        \end{eqnarray}
        Note we have to track the differential in $\vec{u}$ but not in
        $v$ or $\gamma$, which are {\it not} changing.

        And in an orthogonal direction:
        \begin{equation}
          u_y = \frac{u_y'}{\gamma\left(1+v u_x'/c^2\right)}
        \end{equation}
        Differentiating this one:
        \begin{eqnarray}
          {\rm d}u_y  
          &=& \frac{{\rm d}u_y' \left(1 + v u_x'/c^2\right)}
          {\gamma\left(1+v u_x'/c^2\right)^2} -
          \frac{u_y' \left(v/c^2\right) {\rm d}u_x'}{\gamma\left(1+v
            u_x'/c^2\right)^2}\cr
          &=& \frac{1}{\gamma \sigma^2} \left(\sigma {\rm d}u_y'
            - \frac{u_y'  v}{c^2} {\rm d}u_x'\right)
        \end{eqnarray}
        Obviously $z$ will yield a similar answer.

        Now we can see what the acceleration in the unprimed frame
        should be:
        \begin{eqnarray}
          \frac{{\rm d}u_x}{{\rm d}t} &=& \frac{1}{\gamma^3 \sigma^3} 
          \frac{{\rm d}u_x'}{{\rm d}t'} \cr
          \frac{{\rm d}u_y}{{\rm d}t} &=&
          \frac{1}{\gamma^2 \sigma^3} \left(\sigma \frac{{\rm
              d}u_y'}{{\rm d}t'}
            - \frac{u_y'  v}{c^2} \frac{{\rm d}u_x'}{{\rm d}t'}\right)
        \end{eqnarray}
        and just rewriting in accelerations:
        \begin{eqnarray}
          a_x &=& \frac{1}{\gamma^3 \sigma^3} a_x'\cr
          a_y &=&
          \frac{1}{\gamma^2 \sigma^3} \left(\sigma a_y'
            - \frac{u_y'  v}{c^2} a_x' \right)
        \end{eqnarray}
      \end{answer}
      \item If the primed frame is the instantaneous rest frame of the
        particle, show that:
        \begin{eqnarray}
          a_{||}' &=&  \gamma^3 a_{||} \cr
          a_{\perp}' &=&  \gamma^2 a_{\perp}
        \end{eqnarray}

        \begin{answer}
        In this case, $u_x' = u_y' = u_z' = 0$ so $\sigma=1$. The
        parallel direction is $x$ and $y$ can be taken in the
        direction of the perpendicular acceleration. Then the
        equations above fall out directly.
        \end{answer}
      
    \end{enumerate}

\item The essential features of the synchrotron spectrum can be
  derived from the fact that for electrons with high energy ($\gamma
  \gg 1$) the ``pulse shape'' seen as the beamed emission sweeps by is
  defined by a function that depends on angle $\theta$ only in the
  combination $\gamma\theta$. This has the consequence that no matter
  the width of the pulse in time $\Delta t_A$, the shape is always the
  same. That means the spectrum of the emission is characterized
  purely by some critical frequency $\nu_c \sim 1 / \Delta t_A \propto
  \gamma^2$, which has a number of important consequences described in
  class.

  The scattered power can be written as:
  \begin{equation}
    \frac{{\rm d}P}{{\rm d}\Omega} \propto
    \frac{1}{\left(1-\beta\cos\theta\right)^4}
    \left[1 - \frac{\sin^2\theta \cos^2\phi}{\gamma^2 \left(1 -\beta
        \cos\theta\right)^2}\right]
  \end{equation}
  Show in the limit that $\gamma\gg 1$ and for angles ``in the beam''
  $\theta\sim 1/\gamma \ll 1$, that this power is a function of
  $\theta$ only through the combination $\gamma\theta$.

  \begin{answer}
    We begin by recalling the expansion of $\beta$ at large
    velocities:
    \begin{equation}
      \beta = \left(1 + \frac{1}{\gamma^2}\right)^{1/2} \approx 1 -
      \frac{1}{2\gamma^2} 
    \end{equation}
    Then we can also assume $\theta\ll 1$:
    \begin{equation}
      1 - \beta\cos\theta \approx 1-
      \left(1-\frac{1}{2\gamma^2}\right) \left(1 -
      \frac{\theta^2}{2}\right) = 1 - 1 + \frac{1}{2\gamma^2} +
      \frac{\theta^2}{2} - \frac{\theta^2}{4\gamma^2}
    \end{equation}
    For large $\gamma$ and small $\theta$ we can neglect the last
    term so that:
    \begin{equation}
      1 - \beta\cos\theta \approx 
      \frac{1 + \gamma^2\theta^2}{2\gamma^2}
    \end{equation}
    Then we can substitute this into the equation for the power:    
  \begin{eqnarray}
    \frac{{\rm d}P}{{\rm d}\Omega} &\propto&
    \frac{\gamma^8}{\left(1 + \gamma^2\theta^2\right)^4}
    \left[1 - \frac{4 \gamma^4 \theta^2 \cos^2\phi}{\gamma^2 
        \left(1 + \gamma^2\theta^2 \right)^2}\right]\cr
    &\propto&
    \frac{\gamma^8}{\left(1 + \gamma^2\theta^2\right)^6}
    \left[\left(1 + \gamma^2\theta^2\right)^2 - 4 \gamma^2 \theta^2
      \cos^2\phi \right]
  \end{eqnarray}
  That's good enough to answer the question, but it is useful to
  rearrange this into something that separates the orders of the
  expansion better, using $2\cos^2\phi - 1= \cos 2\phi$:
  \begin{equation}
    \frac{{\rm d}P}{{\rm d}\Omega} \propto
    \frac{\gamma^8}{\left(1 + \gamma^2\theta^2\right)^6}
    \left(1 - 2\gamma^2\theta^2 \cos 2\phi + \gamma^4\theta^4\right)
  \end{equation}
  This results shows that in this limit, the shape of the profile is
  set just by the combination $\gamma\theta$, which is key to some of
  the regularity of the properties of synchrotron radiation from a
  single electron (i.e. its spectral shape is just characterized by a
  single frequency defined by $\gamma$ for the electron).
  \end{answer}
  
\end{enumerate}

\end{document}

