\documentclass[11pt, preprint]{article}
\usepackage{hyperref}
\usepackage{rotating}
\usepackage[normalem]{ulem}
\usepackage{environ}
\usepackage{xcolor}
\usepackage{amsmath}

 
\setlength{\footnotesep}{9.6pt}
\setlength{\parskip}{12pt}

\newcounter{thefigs}
\newcommand{\fignum}{\arabic{thefigs}}

\newcounter{thetabs}
\newcommand{\tabnum}{\arabic{thetabs}}

\newcounter{address}

\NewEnviron{answer}[1][]{\color{blue}\expandafter\BODY
{\par \it #1}}

\begin{document}

\begin{center}
  {\bf Radiative Processes in Astrophysics / Problem Set \#8 /
    Answers}
\end{center}

\begin{enumerate}
\item In Shengqi Yang's PhD defense last week she discussed the IR
  lines within the ground configuration of OIII, which is
  1s$^2$2s$^2$2p$^2$. For the IR transitions within the ${}^3$P term,
  she considered, and the plots of the OIII transitions within this
  configuration always show, just the 52 $\mu$m line (between $J=2$
  and $J=1$) and the 88 $\mu$m line (between $J=1$ and $J=0$). But
  there also is a potential transition between $J=2$ and
  $J=0$. Determine what type of transition that third one is,
  i.e. electric dipole, electric quadrupole, or magnetic dipole, and
  explain why in terms of the selection rules. Show how the spacing
  between the three transitions results from Land\'e's interval rule
  for spin-orbit interactions (note that this rule is only {\it
    approximately} true!). If you want to see why the third line
  is usually omitted you can look it up using the line lists on NIST,
  and you will see why Shengqi is safe in ignoring the transition with
  current technology.

  \begin{answer}
    Because it has $\Delta J = 2$, it must be that the third
    transition is an electric quadrupole transition; magnetic and
    electric dipole transitions are not allowed by the selection
    rules.

    Because the states are ordered in energy with $J$, this
    transition has an energy equal to the sum of the energy of the two
    other lines, or in terms of wavelength:
    \begin{equation}
      \frac{1}{\lambda_{20}} = \frac{1}{\lambda_{01}} +
        \frac{1}{\lambda_{12}}
    \end{equation}
    which yields $\lambda_{20} = 32$ $\mu$m.

    Land\'e's rules indicate that the spin-orbit perturbation must be
    proportional to:
    \begin{equation}
      \Delta_{SO} = J(J+1) -  L(L+1) - S(S+1)
    \end{equation}
    and since these are the same term, the differences are just in
    $J$, so from 0 to 2, the differences will be proportional to:
    \begin{eqnarray}
      \Delta_{SO, 0} &=& 0(0+1) -  1(1+1) - 1(1+1) = -4 \cr
      \Delta_{SO, 1} &=& 1(1+1) -  1(1+1) - 1(1+1) = -2 \cr
      \Delta_{SO, 2} &=& 2(2+1) -  1(1+1) - 1(1+1) = +2
    \end{eqnarray}
    which means that $\Delta_{01}$ should be 1/2 of $\Delta_{12}$. This
    is not very exact ($\lambda_{12} = 52$ $\mu$m and not 44 $\mu$m).
    Note that these lines are commonly mislabeled on Grotrian diagrams
    in the literature.
    
    Looking up the transition on NIST, we find $A_{20} \sim 10^{-11}$
    whereas the other two transitions have $A \sim 10^{-6}$ or
    stronger, which explains why this electric quadrupole transition
    may be ignored.
  \end{answer}
\item Also in the same presentation by Dr.~Yang (as she is now!) she
  discussed the variation of the line ratio between 52 $\mu$m and 88
  $\mu$m as a function of electron density, showing a transition
  occurring somewhere in the range $n_e\sim10^2$--$10^3$ cm$^{-3}$. It
  is this variation in the line ratio that allows a constraint on $n_e$.
\begin{enumerate}
  \item Explain qualitatively what is going on---i.e. why is there a
    special density at which this line ratio is likely to change? How
    will that density depend on the $q$ and $A$ values among the three
    levels?

    \begin{answer}
      At high densities, the level populations are kept in
      thermodynamic equilibrium with the electron temperature. This
      means that the line ratio will depend on the Boltzmann ratios
      between the level populations and the Einstein $A$ coefficient.

      At low densities, the upper levels will be collisionally excited
      but will always be radiatively deexcited. So the line ratios
      will not depend at all on the Einstein $A$ coefficient, just on
      the relative excitation rates, which set the level
      populations. (Note that in this case, where the temperature is
      much higher than the energy level differences, the ratio of the
      excitation rates will be close to just the ratio of the
      multiplicities of the states).

      The critical density setting this transition scale will be
      related to the ratio of the deexcitation rates $q_{2j}$ and
      $q_{1j}$ to the radiative decay rates. In detail the definition
      of the critical density is specific to a given level, and is
      given by:
      \begin{equation}
        n_{c,i} = \frac{\sum_{j<i} A_{ij}}{\sum_{i\ne j} q_{ij}}
      \end{equation}
    \end{answer}
  \item Dr.~Wang presented constraints on metallicity and electron
    density; but she did not talk about the temperature of the gas
    (i.e. the electron temperature).  Argue why the relative
    populations of the levels of the ${}^3$P term will not depend on
    temperature, justifying why she didn't talk about it.  (Remember
    OIII only exists in gas that is ionized, either collisionally or
    photoionized).

    \begin{answer}
      If the gas is ionized, the electron temperature is likely to be
      at least of order $10^4$ K. Collisional ionization would
      definitely imply such temperatures (the collisions have to have
      energies $> 10$ eV). Photoionization also implies these high
      temperatures for the electrons, because the typical
      photoionization energies imparted to the electrons are of order
      the ionization energy.

      Thus, as mentioned in the previous answer, the temperatures are
      of order an eV or higher (corresponding to wavelengths of a
      micron), whereas the level splitting we are talking about is a
      few hundredths of an eV (i.e. 50--80 $\mu$m in wavelength).

      So basically all collisions will be energetic enough to excite
      either the $J=1$ or $J=2$ state; their relative populations will
      not be sensitive to temperature.

      If the level splitting energies were of order the gas
      temperature, then the populations of the state $i$ (relative to
      state 0) would be affected strongly by the temperature through a
      factor related to $(g_i / g_0) \exp(-E_{i}/kT)$. Instead, in
      this case it is just $g_i/g_0$ that matters.
    \end{answer}
  \item Write the equations for the balance between the three energy
    states, in terms of the number densities $n_0$, $n_1$, and $n_2$,
    ad in terms of $q_{ij}$ and $A_{ij}$.  You should end up with a
    homogeneous linear system of three equations. I'm not asking you
    to solve the system fully (though it can be done). Also, you can
    leave in factors like $q_{02}$ and $q_{20}$; i.e. you don't need
    to use the relationship between those two rates imposed by
    detailed balance considerations.

    \begin{answer}
      With the left hand side respresenting transitions to each state,
      and right hand side representing the transitions from, we have:
      \begin{eqnarray}
        n_e n_0 q_{02} + n_e n_1 q_{12} &=& n_e n_2 q_{20} + n_e n_2
        q_{21} + n_2 A_{21} \mathrm{\quad for~}J=2 \cr
        n_e n_2 q_{21} + n_e n_0 q_{01} + n_2 A_{21} &=& n_e n_1 q_{12} + n_e n_1
        q_{10} + n_1 A_{10}\mathrm{\quad for~}J=1 \cr
        n_e n_2 q_{20} + n_e n_1 q_{10} + n_1 A_{10} &=& n_e n_0 q_{02} + n_e n_0
        q_{01} \mathrm{\quad for~}J=0
      \end{eqnarray}
      Since there are no terms without $n_0$, $n_1$, or $n_2$ this is
      a homogeneous set of linear equations; as we will see in the
      next answer that means the constraint they impose is on the
      relative populations of $n_1/n_0$ and $n_2/n_0$.
    \end{answer}
  \item Instead, let's think about the low density limit, when
    $n_e\rightarrow 0$. Use the equations for $n_2$ and $n_1$ in order
    to find two equations, one for $n_2/n_0$ and one for $n_1/n_0$, in
    terms of each other. Use the assumption that $n_e$ is small to
    find approximations for $n_2/n_0$ and $n_1/n_0$ to first order in
    $n_e$ (it should be a very simple formula for each!). Finally,
    determine what the relative line flux will be between the $J=2$ to
    $J=1$ vs. the $J=1$ ro $J=0$ transition under these conditions.

    \begin{answer}
      Rearranging the first equation above we find:
      \begin{equation}
        \frac{n_2}{n_0} =  \frac{n_eq_{02} + n_e \frac{n_1}{n_0}
          q_{12}}{n_e\left(q_{20} + q_{21}\right) + A_{21}}
      \end{equation}
      and rearranging the second we find:
      \begin{equation}
        \frac{n_1}{n_0} =  \frac{n_e q_{01} + \frac{n_2}{n_0}
          \left(n_e q_{21}  +A_{21}\right)}
             {n_e\left(q_{12} + q_{10}\right) + A_{10}}
      \end{equation}
      Consider what happens when $n_e q_{ij} \ll A_{ij}$; the
      denominators simply become $A_{21}$ and $A_{10}$. Meanwhile, as
      $n_e$ becomes small the leading order term of $n_1/n_0$ and
      $n_2/n_0$ is of order $n_e$. Then the second term in the
      numerator of the expression for $n_2/n_0$ is of order $n_e^2$,
      and we can approximate:
      \begin{equation}
        \frac{n_2}{n_0} \approx \frac{n_e q_{02}}{A_{21}}
      \end{equation}
      which makes sense as the balance between collisional excitation
      from the most populated state (0) and radiative decay. For
      $n_1/n_0$ we find:
      \begin{equation}
        \frac{n_1}{n_0} \approx \frac{n_e q_{01}}{A_{10}} +
        \frac{1}{A_{10}} \frac{n_2}{n_0} \left(n_e q_{21} + A_{21}\right)
      \end{equation}
      Since $n_2/n_0$ is already of order $n_e$, we can drop the first
      term in the parenthesis. Then using the solution above for
      $n_2/n_0$:
      \begin{equation}
        \frac{n_1}{n_0} \approx \frac{n_e\left(q_{01} +
          q_{02}\right)}{A_{10}}
      \end{equation}
      The qualitative interpretation of this result is that for any
      excitation to state $J=1$, it will decay through $A_{10}$ (since
      the time for a deexcitation is much longer than $1/A_{10}$ at
      low density). And for any excitation to state $J=2$, it will
      {\it also} eventually decay through $A_{21}$ and then through
      $A_{10}$ (because the selection rules forbid a direct decay to
      $J=0$). So the balance for $n_1/n_0$ is between any excitation
      to $J=1$ or $J=2$ versus the radiative decay rate.

      Note that this means that the flux ratio between the two lines
      will be:
      \begin{equation}
        \frac{f_{21}}{f_{10}} = \frac{h\nu_{21} A_{21}
          \left(n_e q_{02}/A_{21}\right)}
             {h\nu_{10} A_{10} 
          \left(n_e (q_{01} + q_{02}) /A_{10}\right)} = \frac{\nu_{21}
               q_{02}}{\nu_{10} (q_{01} + q_{02})}
      \end{equation}
      independent of electron density. See Yang \& Lidz (2020),
      Section 5 for numerical values.
    \end{answer}
  \item In the high density limit, we can assume the levels will be
    kept in thermodynamic equilibrium with the electrons. Determine
    the relative line flux in this case (remember part (b), note that
    the multiplicities of each state will come into the calculation,
    and also leave things in terms of $q$ and $A$ when necessary!).

    \begin{answer}
      In the high density limit:
      \begin{equation}
        \frac{n_1}{n_2} = \frac{g_1}{g_2} \exp\left(- \Delta E_{12} / k
        T\right) \approx g_1/g_2
      \end{equation}
      (where $g_1=3$ and $g_2=5$) and then the line flux ratio will be:
      \begin{equation}
        \frac{f_{21}}{f_{10}}  \approx \frac{\nu_{21}
          g_2A_{21}}{\nu_{10} g_1 A_{10}}
      \end{equation}
      Again, see Yang \& Lidz (2020) for the specific numbers!
    \end{answer}
\end{enumerate}
\end{enumerate}

\end{document}

