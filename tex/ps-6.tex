\documentclass[11pt, preprint]{article}
\usepackage{hyperref}
\usepackage{amsmath}
\usepackage{rotating}
 
\setlength{\footnotesep}{9.6pt}
\setlength{\parskip}{12pt}

\newcounter{thefigs}
\newcommand{\fignum}{\arabic{thefigs}}

\newcounter{thetabs}
\newcommand{\tabnum}{\arabic{thetabs}}

\newcounter{address}

\begin{document}

\begin{center}
  {\bf Radiative Processes in Astrophysics / Problem Set \#6 \\
    Due March 31, 2021}
\end{center}

\begin{enumerate}
\item Rybicki \& Lightman problem 4.3. Feel free to use the solutions
  to guide your work, but it is worth working this problem through
  fully!
\item The essential features of the synchrotron spectrum can be
  derived from the fact that for electrons with high energy ($\gamma
  \gg 1$) the ``pulse shape'' seen as the beamed emission sweeps by is
  defined by a function that depends on angle $\theta$ only in the
  combination $\gamma\theta$. This has the consequence that no matter
  the width of the pulse in time $\Delta t_A$, the shape is always the
  same. That means the spectrum of the emission is characterized
  purely by some critical frequency $\nu_c \sim 1 / \Delta t_A \propto
  \gamma^2$, which has a number of important consequences described in
  class.

  The scattered power can be written as:
  \begin{equation}
    \frac{{\rm d}P}{{\rm d}\Omega} \propto
    \frac{1}{\left(1-\beta\cos\theta\right)^4}
    \left[1 - \frac{\sin^2\theta \cos^2\phi}{\gamma^2 \left(1 -\beta
        \cos\theta\right)^2}\right]
  \end{equation}
  Show in the limit that $\gamma\gg 1$ and for angles ``in the beam''
  $\theta\sim 1/\gamma \ll 1$, that this power is a function of
  $\theta$ only through the combination $\gamma\theta$.
  
\end{enumerate}

\end{document}

