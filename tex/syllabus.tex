\documentclass[12pt]{article}
\usepackage{hyperref}

\setlength{\footnotesep}{9.6pt}
\setlength{\parskip}{12pt}
\setlength{\baselineskip}{15pt}

\newcounter{thefigs}
\newcommand{\fignum}{\arabic{thefigs}}

\newcounter{thetabs}
\newcommand{\tabnum}{\arabic{thetabs}}

\newcounter{address}

\begin{document}

\begin{center}
  {\bf Radiative Processes in Gases / Spring 2021 / Syllabus }
\end{center}


\noindent {\bf Course Description:} Introduction to the radiative
processes relevant to astronomy and astrophysics at the graduate
level, including: energy transfer by radiation; classical and quantum
theory of photon emission; bremsstrahlung; synchrotron radiation;
Compton scattering; plasma effects; and atomic and molecular
electromagnetic transitions.  We will refer to applications in current
astrophysical research.

\noindent {\bf Learning Outcomes:} Broad knowledge of radiative
emission, absorption, and transfer effects and ability to perform
theoretical calculations and estimates of these effects.

\noindent {\bf Assignments:} There will be roughly weekly homeworks
posted on the course web site.  You will submit answers in the form of
a LaTeX file or Python notebook, emailed to me. The course grade will
be based 90\% on the homework assignments, which will be
weighted equally, and 10\% on class participation.

\noindent {\bf Material:} The required textbook is {\it Radiative
  Processes in Astrophysics}, by Rybicki \& Lightman. Other useful
books are {\it Astronomical Spectroscopy} by Jonathan Tennyson, {\it
  Astrophysics of Gaseous Nebulae and Active Galactic Nuclei} by
Osterbrock and Ferland, {\it The Physics of Astrophysics, Volume
  I: Radiation} by Frank Shu, and {\it Physics of the Interstellar and
  Intergalactic Medium} by Bruce Draine.

\noindent The classes will proceed as shown on the next page.  Class
meets Monday and Wednesday at 3:30pm in Room 802 of 726 Broadway.

\noindent You can find the course notes at the
\href{https://github.com/blanton144/radiative/tree/main/docs/pdf}{course
web site}. 

% If we do have time, the following topics may be covered:
% Cherenkov Radiation, the Wouthuysen-Field Effect, Collisional
% Excitation Cooling, Photoelectric Heater, Ly-$\alpha$ absorption and
% scattering, Radiation-driven Winds, or others.

\baselineskip 0pt
\begin{table}
\footnotesize
\begin{tabular}{|c||c|c|}
\hline
{\it Feb.~1} & Radiative Quantities (RL 1.1--1.3) & \cr
{\it Feb.~3} & Radiative Transport (RL 1.4) & \cr
{\it Feb.~8} & Thermal Radiation (RL 1.5) & \cr
{\it Feb.~10} & Einstein Coefficients (RL 1.6) & \cr
{\it Feb.~15} & {\bf NO CLASS} & \cr
{\it Feb.~17} & Scattering (RL 1.7--1.8) & {\bf Exercise \#1 due} \cr
{\it Feb.~18} & E\&M Review (RL 2) & \cr
{\it Feb.~22} & Radiation (RL 3.1--3.3) & {\bf
  Exercise \#2 due} \cr
{\it Feb.~24} & Radiation (RL 3.4) &\cr
{\it Mar.~1} & Radiation (RL 3.5--3.6) &  {\bf Exercise \#3 due} \cr
{\it Mar.~3} & Line Broadening (RL 10.6) & \cr
{\it Mar.~8} & Bremsstrahlung (RL 5.1) & {\bf Exercise \#4 due}\cr
{\it Mar.~10} & Bremsstrahlung (RL 5.2--5.3) &  \cr
{\it Mar.~15} & Synchrotron (RL 6.1, some of RL 4) & \cr
{\it Mar.~17} & Synchrotron (RL 6.2--6.3) & \cr
{\it Mar.~22} & Synchrotron (RL 6.5, 6.8) & {\bf Exercise \#5
  due} \cr
{\it Mar.~24} & Compton Scattering (RL 7) & \cr
{\it Mar.~29} & Compton Scattering (RL 7) & \cr
{\it Mar.~31} & Plasma Effects (RL 8.1-8.2) & {\bf Exercise \#6
  due} \cr
{\it Apr.~5} & Cherenkov Radiation (RL 8.3) & \cr
{\it Apr.~7} & Atomic Structure Basics (RL 9.1--9.2) & {\bf Exercise
  \#7 due}\cr
{\it Apr.~12} & Atomic Structure (Many Electrons) (RL 9.3) & \cr
{\it Apr.~14} & Atomic Transitions (Selection Rules) (RL 10.1--10.4) & {\bf Exercise
  \#8 due}\cr
{\it Apr.~19} & {\bf NO CLASS} & \cr
{\it Apr.~21} & Ionization \& Recombination (RL 10.5) &\cr
{\it Apr.~26} & Molecular Structure (RL 11) &  {\bf Exercise \#9 due} \cr
{\it Apr.~28} & Molecular Transitions (RL 11) & \cr
{\it May~3} & Masers  & {\bf Exercise \#10 due} \cr
{\it May~5} & Dust Absorption \& Scattering &  \cr
{\it May~10} & Dust Heating and Emission & {\bf Exercise \#11 due} \cr
\hline
\end{tabular}
\end{table}

\end{document}

