\documentclass[11pt, preprint]{aastex}
\usepackage{hyperref}
\usepackage{rotating}
 
\setlength{\footnotesep}{9.6pt}

\newcounter{thefigs}
\newcommand{\fignum}{\arabic{thefigs}}

\newcounter{thetabs}
\newcommand{\tabnum}{\arabic{thetabs}}

\newcounter{address}

\begin{document}

\title{\bf Radiative Processes in Astrophysics / Spring 2021 / Syllabus }

\noindent Introduction to the radiative processes relevant to astronomy and
astrophysics at the graduate level, including: energy transfer by
radiation; classical and quantum theory of photon emission;
bremsstrahlung; synchrotron radiation; Compton scattering; plasma
effects; and atomic and molecular electromagnetic transitions.  We
will refer to applications in current astrophysical research.

\noindent You can find the course notes at the
\href{http://blanton144.github.io/radiative}{course web site}. 

\noindent The lectures are based on {\it Radiative Processes in
  Astrophysics}, by Rybicki \& Lightman. Other useful books are {\it
  Astrophysics of Gaseous Nebulae and Active Galactic Nuclei} by
Osterbrock and Ferland, and {\it The Physics of Astrophysics, Volume
  I: Radiation} by Frank Shu. 

\noindent Class meets Monday and Wednesday at 3:30pm in Room 1045 of
726 Broadway, according to Albert.

\noindent The classes will proceed as shown on the next page (subject
to revision!).

\noindent 
Homework will be based on exercises posted on the course web site.
You will submit an answer in the form of a LaTeX file or Python
notebook, emailed to me.

\baselineskip 0pt
\begin{table}
\footnotesize
\begin{tabular}{|c||c|c|}
\hline
{\it Feb.~1} & Radiative Quantities (RL 1.1--1.3) & \cr
{\it Feb.~3} & Radiative Transport (RL 1.4) & \cr
{\it Feb.~8} & Radiative Transport (RL 1.5--1.8) & \cr
{\it Feb.~10} & E\&M Review (RL 2) & {\bf Exercise \#1 due} \cr
{\it Feb.~15} & {\bf NO CLASS} & \cr
{\it Feb.~17} & Radiation (RL 3.1--3.3) & {\bf Exercise \#2 due} \cr
{\it Feb.~18} & Radiation (RL 3.4--3.6) & \cr
{\it Feb.~22} & Emission from Relativistic Particles (RL 4) & \cr
{\it Feb.~24} & Bremsstrahlung (RL 5) & {\bf Exercise \#3 due} \cr
{\it Mar.~1} & Synchrotron (RL 6.1--6.3) & \cr
{\it Mar.~3} & Synchrotron (RL 6.4--6.6) & {\bf Exercise \#4 due}\cr
{\it Mar.~8} & Synchrotron (RL 6.7--6.9) & \cr
{\it Mar.~10} & Compton Scattering (RL 7.1--7.3) &  {\bf Exercise \#5 due} \cr
{\it Mar.~15} & Compton Scattering (RL 7.4--7.7) & \cr
{\it Mar.~17} & Plasma Effects (RL 8) & {\bf Exercise \#6 due} \cr
{\it Mar.~22} & Atomic Structure (RL 9.1--9.3) & \cr
{\it Mar.~24} & Atomic Structure (RL 9.4--9.5)  {\bf Exercise \#6 due} \cr
{\it Mar.~29} & Radiative Transitions (RL 10.1--10.5) & \cr
{\it Mar.~31} & Line Broadening (RL 10.6) & {\bf Exercise \#7 due} \cr
{\it Apr.~5} & Molecular Structure (RL 11.1--11.3) & \cr
{\it Apr.~7} & Molecular Structure (RL 11.4--11.5) & {\bf Exercise \#8 due} \cr
{\it Apr.~12} & Masers &  \cr
{\it Apr.~14} & Dust Absorption &  {\bf Exercise \#9 due}   \cr
{\it Apr.~19} & Dust Scattering & \cr
{\it Apr.~21} & TBD & {\bf Exercise \#10 due} \cr
{\it Apr.~26} & TBD & \cr
{\it Apr.~28} & TBD & {\bf Exercise \#11 due} \cr
{\it May~3} & TBD & \cr
{\it May~5} & TBD & {\bf Exercise \#12 due} \cr
{\it May~10} & TBD & \cr
\hline
\end{tabular}
\end{table}

This is the first year of this course so there is plenty of padding at
the end---I expect that many of these topics will take longer than
planned. If we do have time, the following topics may be covered:
Cherenkov Radiation, the Wouthuysen-Field Effect, Collisional
Excitation Cooling, Photoelectric Heater, Ly-$\alpha$ absorption and
scattering, or others.

\end{document}

